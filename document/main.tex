\documentclass{beamer}

% german content
\usepackage[ngerman]{babel}

%for Lisas PC
\usepackage[T1]{fontenc}
\usepackage[latin1]{inputenc}

% bibliography
\usepackage[
backend=biber,
style=authoryear,
citestyle=authoryear,
autocite=footnote
]{biblatex}
\addbibresource{bibliography.bib}

% images
\usepackage{graphicx}
\graphicspath{ {images/} }

\title{Import/Export von saisonalen Produkten}
% \subtitle{}
\author[Dao, Gabele, Neidel, Neuthor, Spannbauer, Wiegandt]{
  Dao, Duc Trung \and\\
  Gabele, Jörg \and\\
  Neidel, Jonathan \and\\
  Neuthor, Marco \and\\
  Spannbauer, Daniel \and\\
  Wiegandt, Lisa-Marlen
}
\date{Januar 2021}
\institute{HTW Berlin, Angewandte Informatik}
\logo{\includegraphics[width=1cm]{logo}}

% theme + color theme
\usetheme{Szeged}
\usecolortheme{whale}
% see: https://deic-web.uab.cat/~iblanes/beamer_gallery/index.html

\begin{document}
\frame{\titlepage}

\begin{frame}
\frametitle{Exposé}

	Der Import und Export (zur Vereinfachung im Folgenden als Außenhandel benannt) von saisonellen Produkten schwankt je nach Saison des Produktes.

\end{frame}

% show all section names
\begin{frame}
\frametitle{Inhalt}
\tableofcontents
\end{frame}
% how to exclude a section from toc: https://tex.stackexchange.com/a/66633


\section{Planung der Arbeit und Arbeitsverteilung}

\begin{frame}
	\begin{center}
	{\Huge Planung der Arbeit und Arbeitsverteilung}
	\end{center}
\end{frame}

\begin{frame}
	\frametitle{Erstellung eines Zeitplanes}
	@Jonathans Teil
\end{frame}

\begin{frame}
	\frametitle{Projektphasen}
\end{frame}
\begin{frame}
	\frametitle{Rollenverteilung}
	\begin{description}
		\item[Duc Trung Dao, 575477]Main Data Seeker and Preprocessor
		\item[Jörg Gabele, 571127]Master of Theorie
		\item[Jonathan Neidel, 573619]Projekt Manager
		\item[Marco Neuthor, 573738]Principal Visualizer
		\item[Daniel Spannbauer, 572836]Head of Data Processing
		\item[Lisa-Marlen Wiegandt, 572770]Editor of Chief
	\end{description}
    \end{frame}

\section{Definitionen für diese Arbeit}
\begin{frame}
\frametitle{Export und Import}
	\begin{description}
		\item[Außenhandel] Großhandel von Deutschland
		\item[Export] Gesamter Export von Deutschland bezogen jeweils auf ein Produkt, falls nicht anders angegeben.
		\item[Import] Gesamter Import von Deutschland bezogen auf ein Produkt, falls nicht anders angegeben.
	\end{description}
\end{frame}

\section{Datenbeschaffung}
\begin{frame}
	\frametitle{Phase 1}
	allgemeine Datensuche: erste Quelle von 
	BMEL Bundesministerium für Ernährung und Landwirtschaft - statistisches Jahrbuch (jährlich veröffentlich)
	kein Datensatz, sondern PDF - Daten manuell erhoben

	Problem: Es fehlten Monate - Datensätze wären unvollständig

	Lösung: Quellenanalyse des Jahrbuch -> Quelle selbst angeschaut 

	%Bilder/Screenshots einfügen
\end{frame}

\begin{frame}
	\frametitle{Phase 2}
	Datensatz gefunden auf destatis.de - StBa statistisches Bundesamt 
	raw data gefunden - Download möglich
	Export in Excel um csv lesbar zu machen und Übersicht zu verschaffen
\end{frame}

\section{Datenaufbereitung}
\begin{frame}
	\frametitle{1. Phase}
	grobe Aufbereitung der csv in Excel (Hinweis: zB leere Zeilen entfernt)
	%Screenshots?
	%achtung: nicht großer Datensatz sagen
\end{frame}
\begin{frame}
	\frametitle{2. Phase}
	Arbeit mit R
	Skripte geschrieben, die Datensatz aus der csv nutzen
	%Screenshots der
\end{frame}

\section{Visualiserung}
\begin{frame}
\frametitle{Sample frame title}
GitHub TEST
\end{frame}
\begin{frame}
\frametitle{Sample frame title}
This is a text in the first frame. This is a text in the first frame. This is a text in the first frame.
\end{frame}

\section{Informationsgewinnung}
\begin{frame}
\frametitle{Sample frame title}
GitHub TEST
\end{frame}
\begin{frame}
\frametitle{Sample frame title}
This is a text in the first frame. This is a text in the first frame. This is a text in the first frame.
\end{frame}

\section{Fazit}
\begin{frame}
	\frametitle{Fazit}
\end{frame}

% bibliography
% \break
% \printbibliography

\end{document}
